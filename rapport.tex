\documentclass{report}

\usepackage[utf8x]{inputenc}  % accents
\usepackage{geometry}         % marges
\usepackage[francais]{babel}  % langue
\usepackage{graphicx}         % images
\usepackage{verbatim}         % texte préformaté


\title{Rapport de Projet Hubosans4++} 
\author{Lucas BOURNEUF \& Thomas HUBA}


% BEGIN
\begin{document}
\maketitle


\chapter*{Introduction}
\paragraph*{}
Le projet consiste en la réalisation d'un programme codé en C, permettant à un ou plusieurs utilisateurs de jouer à une variante du puissance 4,
apportant notamment la support de 6 joueurs maximum et 3 types de pièces.
\paragraph*{}
Le programe propose, outre la gestion du jeu lui-même, deux affichages, l'un exploitant le terminal, l'autre la bibliothèque SDL.
\paragraph*{}
Il est possible de faire jouer des IA, selon des niveaux de difficulté différents. Ces IA se basent sur un algorithme minimax pour résoudre le jeu.
\paragraph*{}
De plus, il est également possible de faire intervenir des joueurs en réseau, par le protocle TCP/IP.


\chapter*{Analyse}

\section*{Structures du données}
\paragraph*{}
TOUT DOUX

\section*{Algorithmes}
\paragraph*{}
TOUT DOUX

\section*{Gestion d'erreur}
\paragraph*{}
TOUT DOUX


\chapter*{Conclusion}
\paragraph*{}
TOUT DOUX

\end{document}
% END
