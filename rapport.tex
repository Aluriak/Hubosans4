\documentclass{report}

\usepackage[utf8x]{inputenc}  % accents
\usepackage{geometry}         % marges
\usepackage[francais]{babel}  % langue
\usepackage{graphicx}         % images
\usepackage{verbatim}         % texte préformaté


\title{Rapport de Projet Hubosans4++} 
\author{Lucas BOURNEUF \& Thomas HUBA}


% BEGIN
\begin{document}
\maketitle


\chapter*{Introduction}
\paragraph*{}
Le projet consiste en la réalisation d'un programme codé en C, permettant à un ou plusieurs utilisateurs de jouer à une variante du puissance 4, le hubosans4++,
apportant notamment la support de 2 à 6 joueurs et 3 types de pièces. 
\paragraph*{}
Les règles du jeu sont comparable à un puissance 4, si ce n'est que les pièces alignées ne doivent pas nécessairement être de même type, et qu'à chaque tour le joueur choisis, en plus de la colonne, quel type de pièce lâcher.
\paragraph*{}
Le programe propose, outre la gestion du jeu lui-même, deux affichages, l'un exploitant le terminal, l'autre la bibliothèque SDL.
\paragraph*{}
Il est possible de faire jouer des IA, selon des niveaux de difficulté différents. Ces IA se basent sur un algorithme minimax pour résoudre le jeu.



\chapter*{Analyse}

\section*{Structures du données}
\paragraph*{}
TOUT DOUX

\section*{Algorithmes}
\paragraph*{}
TOUT DOUX

\section*{Gestion d'erreur}
\paragraph*{}
TOUT DOUX


\chapter*{Conclusion}
\paragraph*{}
TOUT DOUX

\end{document}
% END
