\documentclass{beamer}

\usepackage[frenchb]{babel}
\usepackage[T1]{fontenc}
\usepackage[utf8]{inputenc}

\usetheme{Warsaw}


%%%%%%%%%%%%%%%%%
% BEGIN         %
%%%%%%%%%%%%%%%%%
\begin{document}

    % % % % % % %
    % FRAME 1   %
    % % % % % % %
    \begin{frame}
    \frametitle{Hubosans4++}
    \framesubtitle{Le puissance 4++}
        Objectif : réaliser un puissance 4 "amélioré", 
            en intégrant notamment les gestions de N joueurs 
            et de trois types de pièces. \\     
        Langage : C, compilé par gcc.
    \end{frame}


    % % % % % % %
    % FRAME 2   %
    % % % % % % %
    \begin{frame}
    \frametitle{Organisation}
    \framesubtitle{du git, du module et du MVC}
        L'organisation s'est faite autour d'un dépôt git, 
            hébergé par github. \\ 
        Le programme est divisé en modules, en essayant 
            de respecter au mieux une architecture MVC.
            % une image, c'est cool, surtout si on peut 
            %   montrer l'archi globale du programme
    \end{frame}


    % % % % % % %
    % FRAME 3   %
    % % % % % % %
    \begin{frame}
    \frametitle{Résolution du problème}
    \framesubtitle{Structures de données}
        De nombreuses structures ont été définies, à commencer 
            par les structures de jeu, de joueurs, et d'action. \\
        \begin{itemize}
        \item[jeu :] plateau de jeu, liste de joueurs, pile d'actions;\\
        \item[joueur :] id, noms, points, \textit{etc};\\
        \item[action :] principal vecteur de communication entre modules, 
            notamment l'affichage et le moteur. Enregistre les actions 
            des joueurs.
        \end{itemize}
    \end{frame}


    % % % % % % %
    % FRAME 4   %
    % % % % % % %
    \begin{frame}
    \frametitle{Résolution du problème}
    \framesubtitle{Découpage fonctionnel}
    \end{frame}


    % % % % % % %
    % FRAME 5   %
    % % % % % % %
    \begin{frame}
    \frametitle{Résolution du problème}
    \framesubtitle{Méthodes de programmation}
    \end{frame}


    % % % % % % %
    % FRAME 6   %
    % % % % % % %
    \begin{frame}
    \frametitle{Avantages et limites}
    \framesubtitle{vers l'infini et au delà}
    \end{frame}


    % % % % % % %
    % FRAME 7   %
    % % % % % % %
    \begin{frame}
    \frametitle{Bilan}
    \framesubtitle{return 0;}
    \end{frame}



%%%%%%%%%%%%%%%%%
% END           %
%%%%%%%%%%%%%%%%%
\end{document}




