\documentclass{beamer}

\usepackage[frenchb]{babel}
\usepackage[T1]{fontenc}
\usepackage[utf8]{inputenc}
\usepackage{graphicx}           % insertion d'images

\usetheme{Warsaw}


%%%%%%%%%%%%%%%%%
% BEGIN         %
%%%%%%%%%%%%%%%%%
\begin{document}

    % % % % % % %
    % FRAME 1   %
    % % % % % % %
    \begin{frame}
    \frametitle{Hubosans4++}
    \framesubtitle{Le puissance 4++}
        Objectif : réaliser un puissance 4 "amélioré", 
            en intégrant notamment les gestions de N joueurs 
            et de trois types de pièces. \\     
        Langage : C, compilé par gcc. \\
        %\includegraphics{ressources/presentation/img1.bitmap}
    \end{frame}


    % % % % % % %
    % FRAME 2   %
    % % % % % % %
    \begin{frame}
    \frametitle{Organisation}
    \framesubtitle{du git, du module et du MVC}
        L'organisation s'est faite autour d'un dépôt git, 
            hébergé par github. \\ 
        Le programme est divisé en modules, en essayant 
            de respecter au mieux une architecture MVC. \\
            % une image, c'est cool, surtout si on peut 
            %   montrer l'archi globale du programme
        Autour du programme, plusieurs fichiers d'intérêt.
            (doc, todo list)
    \end{frame}


    % % % % % % %
    % FRAME 3   %
    % % % % % % %
    \begin{frame}
    \frametitle{Les structures de données}
    \framesubtitle{... sont nos amies}
        De nombreuses structures ont été définies, à commencer 
            par les structures de jeu, de joueur, et d'action, 
            les plus importante et utilisées.\\
        \begin{description}
            \item[jeu :] plateau de jeu, liste de joueurs, pile d'actions;\\
            \item[joueur :] id, noms, points, \textit{etc};\\
            \item[action :] enregistre les actions des joueurs. Communication.\\
        \end{description}
    \end{frame}


    % % % % % % %
    % FRAME 4   %
    % % % % % % %
    \begin{frame}
    \frametitle{Quelques algorithmes parmi les plus notoires}
    \framesubtitle{Détection de puissance 4}
        % TODO
    \end{frame}


    % % % % % % %
    % FRAME 5   %
    % % % % % % %
    \begin{frame}
    \frametitle{Quelques algorithmes parmi les plus notoires}
    \framesubtitle{Minimax, alpha-bêta}
        \begin{block}{Minimax}
            Parcours de l'arbre des possibilités jusqu'à une certaine profondeur pour trouver la meilleure action possible.
        \end{block}
        \begin{block}{Alpha-Bêta}
            Elagage permettant de ne pas parcourir les branches de l'arbre qui ne sont de toute façon pas plus intéressantes que les autres.\\
            Économie de calcul visible, voir nécessaire.
        \end{block}
    \end{frame}


    % % % % % % %
    % FRAME 6   %
    % % % % % % %
    \begin{frame}
    \frametitle{Méthodes de programmation}
    \framesubtitle{vim, NERD* et bidouillages}
        \begin{columns}[c] % structure en colonne
            \begin{column}{5cm} % colonne de gauche
                \begin{block}{vim}
                    Un standard des temps jadis, puissant et personnalisable.
                \end{block}
            \end{column}
            \begin{column}{5cm} % colonne de droite
                \includegraphics[width=5cm, height=3cm]{ressources/presentation/vim.png}
            \end{column}
        \end{columns}
        \begin{exampleblock}{Couplé à des modules d'intérêt}
            fugitive, NERDcommenter, NERDtree, vundle, TeTrIs, taglist... 
        \end{exampleblock}
    \end{frame}


    % % % % % % %
    % FRAME 7   %
    % % % % % % %
    \begin{frame}
    \frametitle{Avantages et limites}
    \framesubtitle{vers l'infini et au delà}
        \begin{columns}[c] % structure en colonne
            \begin{column}{5.5cm} % colonne de gauche
                \begin{exampleblock}{Avantage}
                    % LISTE
                    \begin{itemize}
                        \item modulable, car modulaire;
                        \item code principal de haut niveau;
                    \end{itemize}
                \end{exampleblock}
            \end{column}
            \begin{column}{5.5cm} % colonne de droite
                \begin{alertblock}{Limites}
                    % LISTE
                    \begin{itemize}
                        \item modules trop liés;
                        \item interface grahique pas au top; 
                        \item peu ou pas d'uniformisation générale;
                    \end{itemize}
                \end{alertblock}
            \end{column}
        \end{columns}
    \end{frame}


    % % % % % % %
    % FRAME 8   %
    % % % % % % %
    \begin{frame}
    \frametitle{Bilan}
    \framesubtitle{return 0;}
    \end{frame}



%%%%%%%%%%%%%%%%%
% END           %
%%%%%%%%%%%%%%%%%
\end{document}




